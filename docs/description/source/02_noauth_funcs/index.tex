\chapter{Функционал неавторизованного пользователя}
    \section{Общие элементы страниц}
        \subsection{Фиксированная панель слева}
            \label{sec:baseitems_fixleft_panel}
            \subsubsection{Логотип}
                \begin{enumerate}
                    \item Ссылка с логотипа на главную страницу
                \end{enumerate}
            \subsubsection{Кнопки соцсетей}
                \begin{enumerate}
                    \item Ссылка на группу вконтакте http://vk.com/citizenmoscow
                    \item Ссылка на страницу в facebook https://www.facebook.com/citizenmoscow
                    \item Ссылка на аккаунт в twitter https://twitter.com/citizenmoscow
                    \item Ссылка на страницу в instagram https://instagram.com/citizenmoscow/
                \end{enumerate}
            \subsubsection{Кнопка <<вверх>>}
                \begin{enumerate}
                    \item На главной и в разделах каталога при нажатии перематывает страницы к фильтру. 
                    \item На вторичных страницах перематывает страницу к началу(к главному меню)
                \end{enumerate}
        \subsection{Главное меню}
            \label{sec:baseitems_main_meu}
            \subsubsection{Ссылка на каталог}
                \begin{enumerate}
                    \item Адрес ссылки /catalog/
                    \item Ссылка активна на всех разделах каталога /catalog/*
                \end{enumerate}
            \subsubsection{Ссылка на правила.}
                \begin{enumerate}
                    \item Адрес ссылки /rules/
                    \item С /faq/ идет редирект на /rules/faq/
                    \item Ссылка активна на всех разделах каталога /rules/*
                \end{enumerate}
            
            \subsubsection{Форма поиска}
                \begin{enumerate}
                    \item Placeholder <<поиск>>
                    \item Отправка запроса на поиск по нажатию <<Enter>>
                    \item Переход на страницу поиска (\ref{search_page}) при отправке формы
                \end{enumerate}
        \subsection{Важные ссылки}
            \label{sec:baseitems_important_links}
            \subsubsection{О проекте}
                \begin{enumerate}
                    \item Ссылка указывает на страницу <<как это работает>> (\ref{sec:rules_hiw})
                \end{enumerate}
            \subsubsection{Часто задаваемые вопросы}
                \begin{enumerate}
                    \item Ссылка указывает на страницу <<faq>> (\ref{page_faq})
                \end{enumerate}
            \subsubsection{Обратная связь.}
                \begin{enumerate}
                    \item Ссылка вызывает модальное окно-форму обратной связи без перезагрузки страницы
                    \item Форма обратной связи скрывается кликом на кнопку <<отмена>> или по нажатию <<Esc>>
                    \item Форма имеет два поля для заполнения <<тип обращения>> и <<сообщение>>.
                    \item Сообщение не отправляется, если не заполнено поле <<сообщение>>, выводится сообщение с просьбой заполнить это поле
                    \item Сообщение не отправляется, если не выбрано значение в поле <<тип обращения>>, выводится сообщение с просьбой заполнить это поле
                    \item Доступные типы обращений: <<другое>>.
                    \item После успешной отправки выводится сообщение об успехе и модальное окно с формой закрывается
                    \item После не успешной отправки выводится сообщение об ошибке и модальное окно с формой остается на месте
                \end{enumerate}

        \subsection{Копирайт}
            \label{sec:baseitems_footer_copiright}
            \begin{enumerate}
                \item Текст <<(C) 2016, Активный гражданин, Все права защищены. Публичная оферта>>
                \item Ссылка с <<Публичная оферта>> на <<faq>> (\ref{page_faq})
            \end{enumerate}
        \subsection{Меню разделов магазина}
            \begin{enumerate}
                \item В меню разделов магазина входят только те разделы, которые имеют товары со свойством <<Включено>>(ACTIVE)
                \item Каждый из пунктов имеет ссылку страницу раздела каталога (2.3) вида /catalog/<траслит имени раздела>/
                \item Пункт меню активен при нахождении на соответствующей странице раздела каталога (\ref{page_catalog_section})
            \end{enumerate}
        \subsection{Фильтр по цене и интересам}
            \label{sec:baseitems_interest_filter}
        \subsection{Ссылки фильтра по свойствам}
            \label{sec:baseitems_props_filter}
        \subsection{Ссылки сортировки списка тизеров товаров}
            \label{sec:baseitems_goods_sort}
        \subsection{Область вывода тизеров товаров}
            \label{sec:baseitems_goods_area}
        \subsection{Кнопка подгрузки.}
            \label{sec:baseitems_goods_more}
        \subsection{Тизер товара}




    \section{Главная страница}

        \begin{enumerate}
            \item
            Содержит следующие элементы
            \begin{itemize}
                \item Фиксированная панель слева (см. \ref{baseitems_fixleft_panel})
                \item Главное меню (см. \ref{baseitems_main_meu})
                \item Банер-слайдер (см. \ref{slider})
                \item Фильтр по цене и интересам (см. \ref{baseitems_interest_filter})
                \item Ссылки фильтра по свойствам (см. \ref{baseitems_props_filter})
                \item Ссылки сортировки списка тизеров товаров (см. \ref{baseitems_goods_sort})
                \item Область вывода тизеров товаров (см. \ref{baseitems_goods_area})
                \item Кнопка подгрузки.  (см. \ref{baseitems_goods_more})
                \item Важные ссылки (см. \ref{baseitems_important_links})
                \item Копирайт (см. \ref{baseitems_footer_copiright})
            \end{itemize}
        \end{enumerate}
    
        \subsection{Банер-слайдер}
            \label{sec:slider}
            
            \begin{enumerate}
                \item В слайдере крутятся 2 вида слайдов: слайды-картинки и слайды-товары
                \item Тип слайда задаётся в CMS
            \end{enumerate}
            
            \subsubsection{Слайды-картинки}
                \paragraph{Изображение}
                    \begin{enumerate}
                        \item Картинка для отображения загружается в CMS
                        \item Картинка должна быть с соотношением сторон 87:50
                        \item Если при ином соотношении сторон картинка выводится таким образом, чтобы полностью заполнять слайд
                    \end{enumerate}
                
                \paragraph{Ссылка}

                    \begin{enumerate}
                        \item Ссылка может вести как на внутреннюю страницу магазина, так и на внешний ресурс
                        \item Ссылка открывается в новом окне
                    \end{enumerate}
               
            \subsubsection{Слайды-товары}
                \paragraph{Изображение}
                    \begin{enumerate}
                        \item В качестве ссылки на изображение берётся ссылка на главное фото товара
                    \end{enumerate}
                \paragraph{Ссылка}
                    \begin{enumerate}
                        \item Ссылка ведёт на карточку товара (см. \ref{sec:goods_cart})
                        \item Ссылка открывается в новом окне
                    \end{enumerate}
                \paragraph{Цена}
                \paragraph{Хит}
                \paragraph{Новое}
                \paragraph{Акция}
                \paragraph{Описание}
                \paragraph{Раздел}
                \paragraph{Ссылка с количеством желающих}
            

    \section{Страницы разделов каталога}
        \label{sec:page_catalog_section}


    \section{Карточка товара}
        \label{sec:goods_cart}
        \subsection{Активное фото}
            \subsubsection{Иконка <<Новое>>}
            \subsubsection{Иконка <<Хит>>}
            \subsubsection{Иконка <<Акция>>}
            \subsubsection{Цена в баллах}
            \subsubsection{Число желающих.}

        \subsection{Панель навигации по фотографиям товара}
        \subsection{Средняя оценка товара}
        \subsection{Сообщение <<необходимо набрать N баллов>>}
        \subsection{Название товара}
        \subsection{Артикул}
        \subsection{Описание}
        \subsection{Сообщение <<срок действия вашего заказа>>}
        \subsection{Сообщение <<использовать до>>}
        \subsection{Отзывы.}

    \section{Правила}
        \subsection{Как это работает}
            \label{sec:rules_hiw}
        \subsection{Адреса}
        \subsection{FAQ}
            \label{page_faq}

    \section{Страница поиска}
        \label{search_page}
