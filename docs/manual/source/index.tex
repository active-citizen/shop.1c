% Этот шаблон документа разработан в 2014 году
% Данилом Фёдоровых (danil@fedorovykh.ru) 
% для использования в курсе 
% <<Документы и презентации в \LaTeX>>, записанном НИУ ВШЭ
% для Coursera.org: http://coursera.org/course/latex .
% Исходная версия шаблона --- 
% https://www.writelatex.com/coursera/latex/3.2

\documentclass[a4paper,12pt]{report}

%%% Работа с русским языком
\usepackage{cmap}					% поиск в PDF
\usepackage{mathtext} 				% русские буквы в фомулах
\usepackage[T2A]{fontenc}			% кодировка
\usepackage[utf8]{inputenc}			% кодировка исходного текста
\usepackage[english,russian]{babel}	% локализация и переносы

\usepackage{gloss}

%%% Дополнительная работа с математикой
\usepackage{amsmath,amsfonts,amssymb,amsthm,mathtools} % AMS
\usepackage{icomma} % "Умная" запятая: $0,2$ --- число, $0, 2$ --- перечисление

%% Номера формул
%\mathtoolsset{showonlyrefs=true} % Показывать номера только у тех формул, на которые есть \eqref{} в тексте.
%\usepackage{leqno} % Немуреация формул слева

%% Свои команды
\DeclareMathOperator{\sgn}{\mathop{sgn}}

%% Перенос знаков в формулах (по Львовскому)
\newcommand*{\hm}[1]{#1\nobreak\discretionary{}
{\hbox{$\mathsurround=0pt #1$}}{}}

%%% Работа с картинками
\usepackage{graphicx}  % Для вставки рисунков
\graphicspath{{images/}{images2/}}  % папки с картинками
\setlength\fboxsep{3pt} % Отступ рамки \fbox{} от рисунка
\setlength\fboxrule{1pt} % Толщина линий рамки \fbox{}
\usepackage{wrapfig} % Обтекание рисунков текстом

%%% Работа с таблицами
\usepackage{array,tabularx,tabulary,booktabs} % Дополнительная работа с таблицами
\usepackage{longtable}  % Длинные таблицы
\usepackage{multirow} % Слияние строк в таблице

%%% Теоремы
\theoremstyle{plain} % Это стиль по умолчанию, его можно не переопределять.
\newtheorem{theorem}{Теорема}[section]
\newtheorem{proposition}[theorem]{Утверждение}
 
\theoremstyle{definition} % "Определение"
\newtheorem{corollary}{Следствие}[theorem]
\newtheorem{problem}{Задача}[section]
 
\theoremstyle{remark} % "Примечание"
\newtheorem*{nonum}{Решение}

%%% Программирование
\usepackage{etoolbox} % логические операторы

%%% Страница
\usepackage{extsizes} % Возможность сделать 14-й шрифт
\usepackage{geometry} % Простой способ задавать поля
	\geometry{top=20mm}
	\geometry{bottom=15mm}
	\geometry{left=20mm}
	\geometry{right=15mm}

%	\geometry{top=25mm}
%	\geometry{bottom=35mm}
%	\geometry{left=35mm}
%	\geometry{right=20mm}

 %

\usepackage{setspace} % Интерлиньяж
%\onehalfspacing % Интерлиньяж 1.5
%\doublespacing % Интерлиньяж 2
%\singlespacing % Интерлиньяж 1

\usepackage{lastpage} % Узнать, сколько всего страниц в документе.

\usepackage{soul} % Модификаторы начертания

\usepackage{hyperref}
\usepackage[usenames,dvipsnames,svgnames,table,rgb]{xcolor}
\hypersetup{				% Гиперссылки
    unicode=true,           % русские буквы в раздела PDF
    pdftitle={Заголовок},   % Заголовок
    pdfauthor={Автор},      % Автор
    pdfsubject={Тема},      % Тема
    pdfcreator={Создатель}, % Создатель
    pdfproducer={Производитель}, % Производитель
    pdfkeywords={keyword1} {key2} {key3}, % Ключевые слова
    colorlinks=true,       	% false: ссылки в рамках; true: цветные ссылки
    linkcolor=red,          % внутренние ссылки
    citecolor=green,        % на библиографию
    filecolor=magenta,      % на файлы
    urlcolor=cyan           % на URL
}

% \renewcommand{\familydefault}{\sfdefault} % Начертание шрифта

\usepackage{multicol} % Несколько колонок

\author{Музыка народная, стихи народные}
\title{Магазин поощрений <<Активный гражданин>>. Описание функционала.}
\date{\today}

\makegloss

\begin{document} % конец преамбулы, начало документа

\maketitle
\tableofcontents

\chapter{Типовые сценарии поведения пользователя}

    \section{Сценарии неавторизованного пользователя}

        \subsection{Выход из авторизации.}
        \label{sec:usecase_exit}
            \begin{enumerate}
                \item Перейти на \gloss{AG} 
                \item Совершить выход.
                \item Вернуться на \gloss{shop}
            \end{enumerate}

    
        \subsection{Выбор поощрения по фильтру на главной}
        
        
        \subsection{Выбор поощрения по фильтру в разделах каталога}
        
        \subsection{Просмотр правил <<как это работает>>}
        
        \subsection{Просмотр правил <<Адреса>>}
        
        \subsection{Просмотр правил <<FAQ>>}
        
        \subsection{Написание сообщения в форме обратной связи.}

     \section{Сценарии авторизованного пользователя}

        \subsection{Авторизация.}
            \begin{enumerate}
                \item Перейти на \gloss{AG} 
                \item Совершить вход.
                \item Вернуться на \gloss{shop}
            \end{enumerate}
            
        \subsection{Заказ поощрения}
        \subsection{Размещение отзыва с оценкой.}
        \subsection{Просмотр отзывов.}
        \subsection{Просмотр своих заказов.}
        \subsection{Отмена заказа.}
        \subsection{Печать купона для заказанного поощрения.}
        \subsection{Связь с администрацией по вопросу конкретного заказа.}
        \subsection{Просмотр своих баллов.}
        \subsection{Добавление поощрения в желания.}
        \subsection{Просмотр своих желаний.}
        \subsection{Удаление поощрения из желаний.}

    \section{Сценарии редактора контента}

        \subsection{Редактирование правил <<как это работает>>}
        \subsection{Редактирование пункта правил <<FAQ>>}
        \subsection{Добавление пункта правил <<FAQ>>}
        \subsection{Просмотр обращений через \gloss{FBF}}

    \section{Сценарии администратора магазина}

        \subsection{Добавление поиск пользователя в списке}
        \subsection{Добавление пользователя в группу.}
        \subsection{Выгрузка каталога \gloss{UNF} в \gloss{shop}}
        \subsection{Обмен заказами между \gloss{UNF} в \gloss{shop}}


    \section{Комплексные сценарии.}
        
        \subsection{Заказ}





\chapter{Типовые сценарии поведения пользователя}

    \section{Сценарии неавторизованного пользователя}

        \subsection{Выход из авторизации.}
        \label{sec:usecase_exit}
            \begin{enumerate}
                \item Перейти на \gloss{AG} 
                \item Совершить выход.
                \item Вернуться на \gloss{shop}
            \end{enumerate}

    
        \subsection{Выбор поощрения по фильтру на главной}
        
        
        \subsection{Выбор поощрения по фильтру в разделах каталога}
        
        \subsection{Просмотр правил <<как это работает>>}
        
        \subsection{Просмотр правил <<Адреса>>}
        
        \subsection{Просмотр правил <<FAQ>>}
        
        \subsection{Написание сообщения в форме обратной связи.}

     \section{Сценарии авторизованного пользователя}

        \subsection{Авторизация.}
            \begin{enumerate}
                \item Перейти на \gloss{AG} 
                \item Совершить вход.
                \item Вернуться на \gloss{shop}
            \end{enumerate}
            
        \subsection{Заказ поощрения}
        \subsection{Размещение отзыва с оценкой.}
        \subsection{Просмотр отзывов.}
        \subsection{Просмотр своих заказов.}
        \subsection{Отмена заказа.}
        \subsection{Печать купона для заказанного поощрения.}
        \subsection{Связь с администрацией по вопросу конкретного заказа.}
        \subsection{Просмотр своих баллов.}
        \subsection{Добавление поощрения в желания.}
        \subsection{Просмотр своих желаний.}
        \subsection{Удаление поощрения из желаний.}

    \section{Сценарии редактора контента}

        \subsection{Редактирование правил <<как это работает>>}
        \subsection{Редактирование пункта правил <<FAQ>>}
        \subsection{Добавление пункта правил <<FAQ>>}
        \subsection{Просмотр обращений через \gloss{FBF}}

    \section{Сценарии администратора магазина}

        \subsection{Добавление поиск пользователя в списке}
        \subsection{Добавление пользователя в группу.}
        \subsection{Выгрузка каталога \gloss{UNF} в \gloss{shop}}
        \subsection{Обмен заказами между \gloss{UNF} в \gloss{shop}}


    \section{Комплексные сценарии.}
        
        \subsection{Заказ}





\chapter{Типовые сценарии поведения пользователя}

    \section{Сценарии неавторизованного пользователя}

        \subsection{Выход из авторизации.}
        \label{sec:usecase_exit}
            \begin{enumerate}
                \item Перейти на \gloss{AG} 
                \item Совершить выход.
                \item Вернуться на \gloss{shop}
            \end{enumerate}

    
        \subsection{Выбор поощрения по фильтру на главной}
        
        
        \subsection{Выбор поощрения по фильтру в разделах каталога}
        
        \subsection{Просмотр правил <<как это работает>>}
        
        \subsection{Просмотр правил <<Адреса>>}
        
        \subsection{Просмотр правил <<FAQ>>}
        
        \subsection{Написание сообщения в форме обратной связи.}

     \section{Сценарии авторизованного пользователя}

        \subsection{Авторизация.}
            \begin{enumerate}
                \item Перейти на \gloss{AG} 
                \item Совершить вход.
                \item Вернуться на \gloss{shop}
            \end{enumerate}
            
        \subsection{Заказ поощрения}
        \subsection{Размещение отзыва с оценкой.}
        \subsection{Просмотр отзывов.}
        \subsection{Просмотр своих заказов.}
        \subsection{Отмена заказа.}
        \subsection{Печать купона для заказанного поощрения.}
        \subsection{Связь с администрацией по вопросу конкретного заказа.}
        \subsection{Просмотр своих баллов.}
        \subsection{Добавление поощрения в желания.}
        \subsection{Просмотр своих желаний.}
        \subsection{Удаление поощрения из желаний.}

    \section{Сценарии редактора контента}

        \subsection{Редактирование правил <<как это работает>>}
        \subsection{Редактирование пункта правил <<FAQ>>}
        \subsection{Добавление пункта правил <<FAQ>>}
        \subsection{Просмотр обращений через \gloss{FBF}}

    \section{Сценарии администратора магазина}

        \subsection{Добавление поиск пользователя в списке}
        \subsection{Добавление пользователя в группу.}
        \subsection{Выгрузка каталога \gloss{UNF} в \gloss{shop}}
        \subsection{Обмен заказами между \gloss{UNF} в \gloss{shop}}


    \section{Комплексные сценарии.}
        
        \subsection{Заказ}





\chapter{Типовые сценарии поведения пользователя}

    \section{Сценарии неавторизованного пользователя}

        \subsection{Выход из авторизации.}
        \label{sec:usecase_exit}
            \begin{enumerate}
                \item Перейти на \gloss{AG} 
                \item Совершить выход.
                \item Вернуться на \gloss{shop}
            \end{enumerate}

    
        \subsection{Выбор поощрения по фильтру на главной}
        
        
        \subsection{Выбор поощрения по фильтру в разделах каталога}
        
        \subsection{Просмотр правил <<как это работает>>}
        
        \subsection{Просмотр правил <<Адреса>>}
        
        \subsection{Просмотр правил <<FAQ>>}
        
        \subsection{Написание сообщения в форме обратной связи.}

     \section{Сценарии авторизованного пользователя}

        \subsection{Авторизация.}
            \begin{enumerate}
                \item Перейти на \gloss{AG} 
                \item Совершить вход.
                \item Вернуться на \gloss{shop}
            \end{enumerate}
            
        \subsection{Заказ поощрения}
        \subsection{Размещение отзыва с оценкой.}
        \subsection{Просмотр отзывов.}
        \subsection{Просмотр своих заказов.}
        \subsection{Отмена заказа.}
        \subsection{Печать купона для заказанного поощрения.}
        \subsection{Связь с администрацией по вопросу конкретного заказа.}
        \subsection{Просмотр своих баллов.}
        \subsection{Добавление поощрения в желания.}
        \subsection{Просмотр своих желаний.}
        \subsection{Удаление поощрения из желаний.}

    \section{Сценарии редактора контента}

        \subsection{Редактирование правил <<как это работает>>}
        \subsection{Редактирование пункта правил <<FAQ>>}
        \subsection{Добавление пункта правил <<FAQ>>}
        \subsection{Просмотр обращений через \gloss{FBF}}

    \section{Сценарии администратора магазина}

        \subsection{Добавление поиск пользователя в списке}
        \subsection{Добавление пользователя в группу.}
        \subsection{Выгрузка каталога \gloss{UNF} в \gloss{shop}}
        \subsection{Обмен заказами между \gloss{UNF} в \gloss{shop}}


    \section{Комплексные сценарии.}
        
        \subsection{Заказ}





\chapter{Типовые сценарии поведения пользователя}

    \section{Сценарии неавторизованного пользователя}

        \subsection{Выход из авторизации.}
        \label{sec:usecase_exit}
            \begin{enumerate}
                \item Перейти на \gloss{AG} 
                \item Совершить выход.
                \item Вернуться на \gloss{shop}
            \end{enumerate}

    
        \subsection{Выбор поощрения по фильтру на главной}
        
        
        \subsection{Выбор поощрения по фильтру в разделах каталога}
        
        \subsection{Просмотр правил <<как это работает>>}
        
        \subsection{Просмотр правил <<Адреса>>}
        
        \subsection{Просмотр правил <<FAQ>>}
        
        \subsection{Написание сообщения в форме обратной связи.}

     \section{Сценарии авторизованного пользователя}

        \subsection{Авторизация.}
            \begin{enumerate}
                \item Перейти на \gloss{AG} 
                \item Совершить вход.
                \item Вернуться на \gloss{shop}
            \end{enumerate}
            
        \subsection{Заказ поощрения}
        \subsection{Размещение отзыва с оценкой.}
        \subsection{Просмотр отзывов.}
        \subsection{Просмотр своих заказов.}
        \subsection{Отмена заказа.}
        \subsection{Печать купона для заказанного поощрения.}
        \subsection{Связь с администрацией по вопросу конкретного заказа.}
        \subsection{Просмотр своих баллов.}
        \subsection{Добавление поощрения в желания.}
        \subsection{Просмотр своих желаний.}
        \subsection{Удаление поощрения из желаний.}

    \section{Сценарии редактора контента}

        \subsection{Редактирование правил <<как это работает>>}
        \subsection{Редактирование пункта правил <<FAQ>>}
        \subsection{Добавление пункта правил <<FAQ>>}
        \subsection{Просмотр обращений через \gloss{FBF}}

    \section{Сценарии администратора магазина}

        \subsection{Добавление поиск пользователя в списке}
        \subsection{Добавление пользователя в группу.}
        \subsection{Выгрузка каталога \gloss{UNF} в \gloss{shop}}
        \subsection{Обмен заказами между \gloss{UNF} в \gloss{shop}}


    \section{Комплексные сценарии.}
        
        \subsection{Заказ}





\chapter{Типовые сценарии поведения пользователя}

    \section{Сценарии неавторизованного пользователя}

        \subsection{Выход из авторизации.}
        \label{sec:usecase_exit}
            \begin{enumerate}
                \item Перейти на \gloss{AG} 
                \item Совершить выход.
                \item Вернуться на \gloss{shop}
            \end{enumerate}

    
        \subsection{Выбор поощрения по фильтру на главной}
        
        
        \subsection{Выбор поощрения по фильтру в разделах каталога}
        
        \subsection{Просмотр правил <<как это работает>>}
        
        \subsection{Просмотр правил <<Адреса>>}
        
        \subsection{Просмотр правил <<FAQ>>}
        
        \subsection{Написание сообщения в форме обратной связи.}

     \section{Сценарии авторизованного пользователя}

        \subsection{Авторизация.}
            \begin{enumerate}
                \item Перейти на \gloss{AG} 
                \item Совершить вход.
                \item Вернуться на \gloss{shop}
            \end{enumerate}
            
        \subsection{Заказ поощрения}
        \subsection{Размещение отзыва с оценкой.}
        \subsection{Просмотр отзывов.}
        \subsection{Просмотр своих заказов.}
        \subsection{Отмена заказа.}
        \subsection{Печать купона для заказанного поощрения.}
        \subsection{Связь с администрацией по вопросу конкретного заказа.}
        \subsection{Просмотр своих баллов.}
        \subsection{Добавление поощрения в желания.}
        \subsection{Просмотр своих желаний.}
        \subsection{Удаление поощрения из желаний.}

    \section{Сценарии редактора контента}

        \subsection{Редактирование правил <<как это работает>>}
        \subsection{Редактирование пункта правил <<FAQ>>}
        \subsection{Добавление пункта правил <<FAQ>>}
        \subsection{Просмотр обращений через \gloss{FBF}}

    \section{Сценарии администратора магазина}

        \subsection{Добавление поиск пользователя в списке}
        \subsection{Добавление пользователя в группу.}
        \subsection{Выгрузка каталога \gloss{UNF} в \gloss{shop}}
        \subsection{Обмен заказами между \gloss{UNF} в \gloss{shop}}


    \section{Комплексные сценарии.}
        
        \subsection{Заказ}





\chapter{Типовые сценарии поведения пользователя}

    \section{Сценарии неавторизованного пользователя}

        \subsection{Выход из авторизации.}
        \label{sec:usecase_exit}
            \begin{enumerate}
                \item Перейти на \gloss{AG} 
                \item Совершить выход.
                \item Вернуться на \gloss{shop}
            \end{enumerate}

    
        \subsection{Выбор поощрения по фильтру на главной}
        
        
        \subsection{Выбор поощрения по фильтру в разделах каталога}
        
        \subsection{Просмотр правил <<как это работает>>}
        
        \subsection{Просмотр правил <<Адреса>>}
        
        \subsection{Просмотр правил <<FAQ>>}
        
        \subsection{Написание сообщения в форме обратной связи.}

     \section{Сценарии авторизованного пользователя}

        \subsection{Авторизация.}
            \begin{enumerate}
                \item Перейти на \gloss{AG} 
                \item Совершить вход.
                \item Вернуться на \gloss{shop}
            \end{enumerate}
            
        \subsection{Заказ поощрения}
        \subsection{Размещение отзыва с оценкой.}
        \subsection{Просмотр отзывов.}
        \subsection{Просмотр своих заказов.}
        \subsection{Отмена заказа.}
        \subsection{Печать купона для заказанного поощрения.}
        \subsection{Связь с администрацией по вопросу конкретного заказа.}
        \subsection{Просмотр своих баллов.}
        \subsection{Добавление поощрения в желания.}
        \subsection{Просмотр своих желаний.}
        \subsection{Удаление поощрения из желаний.}

    \section{Сценарии редактора контента}

        \subsection{Редактирование правил <<как это работает>>}
        \subsection{Редактирование пункта правил <<FAQ>>}
        \subsection{Добавление пункта правил <<FAQ>>}
        \subsection{Просмотр обращений через \gloss{FBF}}

    \section{Сценарии администратора магазина}

        \subsection{Добавление поиск пользователя в списке}
        \subsection{Добавление пользователя в группу.}
        \subsection{Выгрузка каталога \gloss{UNF} в \gloss{shop}}
        \subsection{Обмен заказами между \gloss{UNF} в \gloss{shop}}


    \section{Комплексные сценарии.}
        
        \subsection{Заказ}





\chapter{Типовые сценарии поведения пользователя}

    \section{Сценарии неавторизованного пользователя}

        \subsection{Выход из авторизации.}
        \label{sec:usecase_exit}
            \begin{enumerate}
                \item Перейти на \gloss{AG} 
                \item Совершить выход.
                \item Вернуться на \gloss{shop}
            \end{enumerate}

    
        \subsection{Выбор поощрения по фильтру на главной}
        
        
        \subsection{Выбор поощрения по фильтру в разделах каталога}
        
        \subsection{Просмотр правил <<как это работает>>}
        
        \subsection{Просмотр правил <<Адреса>>}
        
        \subsection{Просмотр правил <<FAQ>>}
        
        \subsection{Написание сообщения в форме обратной связи.}

     \section{Сценарии авторизованного пользователя}

        \subsection{Авторизация.}
            \begin{enumerate}
                \item Перейти на \gloss{AG} 
                \item Совершить вход.
                \item Вернуться на \gloss{shop}
            \end{enumerate}
            
        \subsection{Заказ поощрения}
        \subsection{Размещение отзыва с оценкой.}
        \subsection{Просмотр отзывов.}
        \subsection{Просмотр своих заказов.}
        \subsection{Отмена заказа.}
        \subsection{Печать купона для заказанного поощрения.}
        \subsection{Связь с администрацией по вопросу конкретного заказа.}
        \subsection{Просмотр своих баллов.}
        \subsection{Добавление поощрения в желания.}
        \subsection{Просмотр своих желаний.}
        \subsection{Удаление поощрения из желаний.}

    \section{Сценарии редактора контента}

        \subsection{Редактирование правил <<как это работает>>}
        \subsection{Редактирование пункта правил <<FAQ>>}
        \subsection{Добавление пункта правил <<FAQ>>}
        \subsection{Просмотр обращений через \gloss{FBF}}

    \section{Сценарии администратора магазина}

        \subsection{Добавление поиск пользователя в списке}
        \subsection{Добавление пользователя в группу.}
        \subsection{Выгрузка каталога \gloss{UNF} в \gloss{shop}}
        \subsection{Обмен заказами между \gloss{UNF} в \gloss{shop}}


    \section{Комплексные сценарии.}
        
        \subsection{Заказ}






\end{document}
