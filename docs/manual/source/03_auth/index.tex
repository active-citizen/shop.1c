\chapter{Авторизация}
    \label{sec:auth}
    
    Через процедуру авторизации проходит любой пользователь, желающий получить 
    доступ к дополнительным функциям, к которым не имеет доступа пользователь группы
    \\\gloss{nonauth_user}. Через неё не проходит только пользователь \\\gloss{admin}.
    
    Авторизация посетителя магазина основана на кроссдоменной передаче сессии от \gloss{AG}.

    \section{Общее в процедуре авторизации(кроссдоменная авторизация)}
    \label{sec:auth_common}
    
        \begin{enumerate}
            \item Пользователь магазина может быть авторизован только если он авторизован на \gloss{AG}.
            \item Если посетитель не авторизован в магазине, то на его странице особым образом формируется 
                ссылка на страницу \gloss{AG}, выводящую информацию об \gloss{AGSESSIONID} в зашифрованном виде.
            \item Расшифровав \gloss{AGSESSIONID}, магазин получает \gloss{AGPROFILE} у \gloss{AGAPI}.
            \item Далее, в зависимости от того, заходил ли посетитель в \gloss{shop} ранее, 
                происходит первичная(см.\ref{sec:auth_first}) или вторичная(см.\ref{sec:auth_second}) авторизация
        \end{enumerate}
    
    \section{Первичная авторизация пользователя}
    \label{sec:auth_first}

        \begin{enumerate}
            \item Производится если существующий пользователь \gloss{AG} впервые зашел в \gloss{shop}.
            \item После того как \gloss{shop} получит \gloss{AGPROFILE} (см. \ref{sec:auth_common}), через \gloss{BXAPI} создаётся \gloss{BXUSER}
            \item В качестве логина для него используется телефон, на который зарегистрирован \gloss{AGPROFILE}.
            \item Пароль задаётся случайно(так как сам по себе \gloss{BXUSER} не преднозначен для самостоятельной авторизации в \gloss{shop})
            \item Далее посетитель автоматически авторизуется на \gloss{shop} и продолжает работу как \gloss{auth_user}
        \end{enumerate}
    
    
    \section{Вторичная авторизация пользователя}
    \label{sec:auth_second}
        \begin{enumerate}
            \item Производится если существующий пользователь \gloss{AG} повторно зашел в \gloss{shop}, и не авторизован в нем как \gloss{BXUSER}.
            \item После того как \gloss{shop} получит \gloss{AGPROFILE} (см. \ref{sec:auth_common}), 
                через \gloss{BXAPI} ищется \gloss{BXUSER} с таким же номером телефона.
            \item Далее посетитель автоматически авторизуется на \gloss{shop} и продолжает работу как \gloss{auth_user}
        \end{enumerate}
    
    
    \section{Получение истории баллов}
    \label{sec:get_points_history}
    \begin{enumerate}
        \item Для принятия решения о возможности совершения покупки в \gloss{shop} 
            необходимо получить информацию о количестве баллов у пользователя \gloss{AG}
        \item История баллов после авторизации получается с помощью \gloss{AGAPI} в виде списка объектов типа
            \gloss{AGTRANSACT} и оформления их в виде объектов типа \gloss{BXTRANSACT}
        \item Затем баланс баллов, полученных из \gloss{AG} используется в процессе овормления заказов \gloss{shop}
    \end{enumerate}

    
    \section{Получение заказов}
    \begin{enumerate}
        \item Для отслеживания состояния заказов объекты типа \gloss{BXORDER} синхронизируются с \gloss{UNF} посредством 
            процесса \gloss{exchange1c}.
        \item Центральным хранилищем информации о заказах является \gloss{UNF}, а \gloss{shop} является как бы дополнительным web-интерфейсом к нему
    \end{enumerate}

    
    
    
    

